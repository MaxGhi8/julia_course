\documentclass{article}

% Language setting
% Replace `english' with e.g. `spanish' to change the document language
\usepackage[english]{babel}

% Set page size and margins
% Replace `letterpaper' with `a4paper' for UK/EU standard size
\usepackage[letterpaper,top=2cm,bottom=2cm,left=3cm,right=3cm,marginparwidth=1.75cm]{geometry}

% Useful packages
\usepackage{amsmath}
\usepackage{graphicx}
\usepackage[colorlinks=true, allcolors=blue]{hyperref}

\title{Introduction to Julia Programming for Data Science and Scientific Computing}
\author{Ivan Bioli, Massimiliano Ghiotto}
\date{}

\begin{document}
\maketitle


\subsection*{Lectures}
Lectures will be weekly (every Wednesday), from 6 November to 29 January from 16:30 to 18:00.
Each lecture will be divided into two: theory and exercises. In the first half of the course, the lecturers will teach Julia the basics. In the second half of the course, participants who wish to do so will be asked to choose an advanced topic that they will present.

\subsection*{Credits}
$3$ CFU ($20$ hours).

\subsection*{Prerequisites}
Basic knowledge of programming.

\subsection*{Objectives}
The objective of this course is to provide students with a foundational understanding of the Julia programming language. By the end of the course, students will be able to efficiently use Julia to solve computational problems in different areas.

\subsection*{Course Overview}
This \textbf{reading course} introduces students to Julia, a high-performance programming language designed for technical computing, data science, and machine learning. It covers the fundamentals of Julia, focusing on its syntax, core features, and powerful libraries, emphasising practical applications in data analysis, numerical computation, and visualization. By the end of the course, participants will gain a strong understanding of how to leverage Julia's speed and expressiveness for scientific and computational tasks.

\subsection*{Course Structure}
\begin{enumerate}
    \item Introduction to Julia:
          \begin{itemize}
              \item Overview of Julia and its ecosystem;
              \item Basic syntax: variables, types, functions;
              \item Control flow and data structures;
              \item Working with Julia environment.
          \end{itemize}

    \item Data Manipulation and Visualization:
          \begin{itemize}
              \item Introduction to DataFrames.jl: loading, cleaning, and exploring datasets (optional),
              \item Creating visualizations with Plots.jl and Makie.jl.
          \end{itemize}

    \item Numerical and Scientific Computing:
          \begin{itemize}
              \item Arrays, matrices, and linear algebra in Julia;
              \item Solving differential equations;
              \item Machine learning with Flux.jl and MLJ.jl;
              \item Overview of GaussianRandomField.jl library for generating datasets.
          \end{itemize}

    \item Advanced Topics in Julia (TBA, based on the participants interest)
\end{enumerate}

\subsection*{Readings and Resources}
\begin{itemize}
    \item Primary Text: official Julia documentation;
    \item Exercises at Julia's platform \textit{Exercism.org};
\end{itemize}

\end{document}